\documentclass[dvipdfmx]{ampbt}

%% クラスオプション:
%% chapter:   \chapterコマンドを使用可能にする(jsbook (report) を使う).
%% duplexing: 両面印刷用の PDF を出力する.
%% その他 jsclasses に指定可能なオプションが指定できます(そのまま渡される).

%% 報告書の題目 %%%%%%%%%%%%%%%%%%%%%%%%%%%%%%%%%%%%%%%%%%%%%%%%%%%%%%%%%%%%%%%%%
\title{Friend-to-friendネットワークにおける} % 題目1行目
      {効率的な非中央集権的ルーティング}                         % 題目2行目
      {}                                         % 題目3行目
%% 指導教員 %%%%%%%%%%%%%%%%%%%%%%%%%%%%%%%%%%%%%%%%%%%%%%%%%%%%%%%%%%%%%%%%%%%%%
\supervisors{}{講師}    % 指導教員1人目 {氏名}{職名}
            {}{}    % 指導教員2人目 {氏名}{職名}
            {}{}                % 指導教員3人目 {氏名}{職名}
%% 入学年月 %%%%%%%%%%%%%%%%%%%%%%%%%%%%%%%%%%%%%%%%%%%%%%%%%%%%%%%%%%%%%%%%%%%%%
\entrancedate{24}{4}            % {年(平成)}{月}
%% 著者氏名 %%%%%%%%%%%%%%%%%%%%%%%%%%%%%%%%%%%%%%%%%%%%%%%%%%%%%%%%%%%%%%%%%%%%%
\author{髙橋}{彰}             % {姓}{名}
%% 提出日 %%%%%%%%%%%%%%%%%%%%%%%%%%%%%%%%%%%%%%%%%%%%%%%%%%%%%%%%%%%%%%%%%%%%%%%
\submissiondate{29}{1}{XX}      % {年(平成)}{月}{日}
%% 提出年度 %%%%%%%%%%%%%%%%%%%%%%%%%%%%%%%%%%%%%%%%%%%%%%%%%%%%%%%%%%%%%%%%%%%%%
\submissionjay{28}              % {年度(平成)}
%% 摘要 %%%%%%%%%%%%%%%%%%%%%%%%%%%%%%%%%%%%%%%%%%%%%%%%%%%%%%%%%%%%%%%%%%%%%%%%%
\abstract{%
  本研究では, F2Fネットワークトポロジーのスモール・ワールド性を利用し, 効率的かつ非中央集権的なルーティングを実現するための手法を提起する. 
}
%% パッケージの読み込みや自分用のマクロの定義 %%%%%%%%%%%%%%%%%%%%%%%%%%%%%%%%%%%
\usepackage{amsmath,amssymb}
\newcommand{\rme}{\mathrm{e}}

%% 出力の制御 %%%%%%%%%%%%%%%%%%%%%%%%%%%%%%%%%%%%%%%%%%%%%%%%%%%%%%%%%%%%%%%%%%%

%% 本文を出力しない場合,次の行をコメントアウトして下さい.
% \outputbodyfalse

%% 末尾に表紙,背表紙を出力しない場合,次の行をコメントアウトして下さい.
% \outputcoverfalse

%% 末尾に提出用摘要を出力しない場合,次の行をコメントアウトして下さい.
% \outputabstractforsubmissionfalse

%% ampbt.cls では表紙等の作成のために geometry パッケージを使用しているため,本文
%% のレイアウトを変えるために \usepackage[...]{geometry} とすると Option clash が
%% 発生します.何らかの理由で本文のレイアウトを変更したい場合は \geometry{...} を
%% 使用して下さい.
%% また,jsclasses を使用しているため,例えば 3cm を指定したい場合は 3truecm と書
%% く必要があります.
% \geometry{hmargin=3truecm,vmargin=2truecm}

\begin{document}
\ifoutputbody
%% 中表紙,摘要,目次 %%%%%%%%%%%%%%%%%%%%%%%%%%%%%%%%%%%%%%%%%%%%%%%%%%%%%%%%%%%
\makeinsidecover                % 中表紙
\makeabstract                   % 摘要
\maketoc                        % 目次
\setcounter{page}{1}            % 本文のページ番号を1から始める
%% 本文 %%%%%%%%%%%%%%%%%%%%%%%%%%%%%%%%%%%%%%%%%%%%%%%%%%%%%%%%%%%%%%%%%%%%%%%%%
\section{序論}
\cite{clarke2001freenet}
\cite{sandberg2006distributed}
\cite{sandberg2006evolution}
\cite{sandberg2008neighbor}
\cite{mogren2008adaptive}
\cite{evans2007routing}
\cite{clarke2010private}
\cite{schiller2011attack}
\cite{roos2012provable}
\cite{roos2013contribution}
\cite{roos2016dealing}
\cite{hofer2013greedy}
\cite{roos2016anonymous}
\cite{roos2016analyzing}
\cite{kleinberg2000small}
\cite{csimcsek2008navigating}
\cite{serrano2008self}
\cite{boguna2009navigating}
\cite{boguna2009navigability}
\cite{kleinberg2007geographic}
\cite{cvetkovski2009hyperbolic}
\cite{krioukov2010hyperbolic}
\cite{boguna2010sustaining}
\cite{papadopoulos2015network}
\cite{blasius2016efficient}
\cite{kleinberg2006complex}
\cite{huang2014navigation}

\section{先行研究}
foo
\section{アルゴリズム}
foo
\subsection{Metropolis-Hastingsアルゴリズム}
foo

\subsection{非中央集権的ルーティング}
foo
\section{シミュレーション}
foo
\section{結論}
foo
%% 謝辞 %%%%%%%%%%%%%%%%%%%%%%%%%%%%%%%%%%%%%%%%%%%%%%%%%%%%%%%%%%%%%%%%%%%%%%%%%
\acknowledgment


%% 参考文献 %%%%%%%%%%%%%%%%%%%%%%%%%%%%%%%%%%%%%%%%%%%%%%%%%%%%%%%%%%%%%%%%%%%%%
\clearpage
\addcontentsline{toc}{section}{\refname} % 目次に参考文献を追加する.
                                         % chapter使用時は削除すること.

%% BibTeX 等を用いる場合は,上の thebibliography 環境を消してここに該当コードを
%% 挿入すること.
\bibliographystyle{acm}
\bibliography{refs}

%% 付録 %%%%%%%%%%%%%%%%%%%%%%%%%%%%%%%%%%%%%%%%%%%%%%%%%%%%%%%%%%%%%%%%%%%%%%%%%
%% 付録は不要ならば削除してよい.
\appendix

\section{意味のない付録}

\begin{table}[htbp]
  \caption{これは意味のない表です.}
  \centering
  \begin{tabular}{c|cc}
      &  A  &  B \\
    \hline
    C &  70 & 80 \\
    D & 100 &  0
  \end{tabular}
\end{table}

%% 本文ここまで %%%%%%%%%%%%%%%%%%%%%%%%%%%%%%%%%%%%%%%%%%%%%%%%%%%%%%%%%%%%%%%%%
\fi
\ifoutputcover
\evenclearpage
%% 表紙,背表紙,提出用摘要 %%%%%%%%%%%%%%%%%%%%%%%%%%%%%%%%%%%%%%%%%%%%%%%%%%%%%
\makecover                      % 表紙
\makespine[1]                   % 背表紙([] 内は出力枚数)
\makeinsidecover                % 中表紙
\fi
\ifoutputabstractforsubmission
\makeabstractforsubmission      % 提出用摘要
\fi
\end{document}
